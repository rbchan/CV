\singlespacing
\thispagestyle{plain} % No header on first page

\begin{center}
{\large \bf Richard B. Chandler} \\
%Assistant Professor \\
Warnell School of Forestry and Natural Resources \\ %, %\\
University of Georgia, % \\
180 East Green St., Athens, GA 30602 \\
\href{mailto:rchandler@warnell.uga.edu}{rchandler@warnell.uga.edu},  %\\
%rchandler@warnell.uga.edu \\
706-542-5815 \\
\end{center}


\vspace{0.8cm}

{\large \bf Education} \\
\rule[3mm]{\textwidth}{0.3mm}

\begin{itemize}
    \item \textnormal{University of Massachusetts Amherst}. 2011. Ph.D. Wildlife Conservation. 
    \item \textnormal{University of Massachusetts Amherst}. 2006. M.S. Wildlife Conservation. 
    \item \textnormal{University of Vermont}. 2002. B.S. Wildlife Biology. 
\end{itemize}


\vspace{0.5cm}

{\large \bf Professional Positions} \\
\rule[3mm]{\textwidth}{0.3mm}

\begin{itemize}

\item \textnormal{Associate Professor}, %Warnell School of Forestry and
%  Natural Resources, 
  University of Georgia, 2013--present

\item \textnormal{Postdoctoral Research Ecologist}, USGS Patuxent
  Wildlife Research Center, 2010--2013

\end{itemize}


\vspace{0.5cm}

{\large \bf Publications} \\
\rule[3mm]{\textwidth}{0.3mm}

\emph{Books}
\begin{itemize}

\item Royle, J.A., R.B. Chandler, R. Sollmann, and B. Gardner. 2014. Spatial
Capture-Recapture. Academic Press. 

\end{itemize}

\vspace{0.5cm}

\emph{Journal Articles}

\begin{itemize}

% \item[] {\centering In review \\}





\item[] {\centering In press \\}

\item Hooker, M.J., R.B. Chandler, B.T. Bond, and M.J. Chamberlain. In
press. Assessing population viability of black bears using spatial
capture-recapture models. Journal of Wildlife Management. 

\item Howell, P.E., B.R. Hossack, E. Muths, B.H. Sigafus, and
R.B. Chandler. In press. Informing amphibian conservation efforts
with abundance-based metapopulation models. Herpetologica.

\item[] {\centering 2020 \\}

\item Merker, S. and R.B. Chandler. 2020. Identifying global hotspots
of avian trailing-edge population diversity. Global Ecology and
Conservation 22:e00915.   

\item Howell, P. E., B. R. Hossack, E. Muths, B. H. Sigafus,
  A. Chenevert-Steffler, R. B. Chandler. 2020. A statistical 
  forecasting approach to metapopulation viability
  analysis. Ecological Applications 3:e02038. 
  
\item[] {\centering 2019 \\}

\item Abernathy, H., D. Crawford, E. Garrison, R. Chandler,
  L.M. Conner, K.V. Miller, M.J. Cherry. 2019. Deer movement and
  resource selection during Hurricane Irma -- Implications for
  Wildlife and Extreme Climatic Events. Proceedings of the Royal
  Society B 286:20192230. 
  
\item Bennett, R.E., A.D. Rodewald, K.V. Rosenberg, R. Chandler,
  L. Chavarria-Duriaux, J.A. Gerwin, D.I. King, and
  J. Larkin. 2019. Drivers of variation in migration behavior for a
  linked population of long-distance migratory passerine. The Auk
  136(4): ukz051.
  
\item Jimenez, J., R. Chandler, J. Tobajas, E. Descalzo, R. Mateo,
  P. Ferreras. 2019. Generalized spatial mark–resight models with
  incomplete identification: An application to red fox density
  estimates. Ecology and Evolution 9:4739--4748. 
  
\item Augustine, B., J.A. Royle, S. Murphy, R.B. Chandler,
  J. Cox, M. Kelly. 2019. Spatial capture-recapture for
  categorically marked populations with an application to genetic
  capture-recapture. Ecosphere 10:e02627.

\item Crawford, D.A., M.J. Cherry, B.D. Kelly, E.P. Garrison,
  D. Shindle, L.M. Conner, R.B. Chandler, and
  K.V. Miller. 2019. Chronology of reproductive investment determines
  predation risk aversion in a felid-ungulate system. Ecology and
  Evolution 9:3264--3275. 

\item Coughlin, E.M., B.M. Shamblin, H.R. Tumas, R.B. Chandler, and
  C.J. Nairn. 2019. The complete mitochondiral genome of the
  Canada warbler ({\it Cardellina canadensis}). Mitochondrial DNA Part
  B 4:450--451.

%\newpage
  
\item[] {\centering 2018 \\}

\item Chandler, R.B., K. Engebretsen, M. Cherry, and E. Garrison, 
  K.V. Miller. 2018. Estimating recruitment from
  capture-recapture data by modeling spatio-temporal variation in
  birth and age-specific survival rates. Methods in Ecology and
  Evolution 10:2115--2130.

\item Tumas H.R., B.R. Shamblin, M. Woodrey, N.P. Nibbelink,
  R. Chandler, C.J. Nairn. 2018. Landscape genetics of the
  foundational salt marsh plant species black needlerush ({\it Juncus
    roemerianus Scheele}) across the northeastern Gulf of
  Mexico. Landscape Ecology 33:1585--1601.
  
\item McConnell, M.D., A.P. Monroe, R.B. Chandler, W.E. Palmer,
  S.D. Wellendorf, L.W. Burger, and J.A. Martin. 2018. Factors
  influencing northern bobwhite recruitment with implications for
  population growth. Auk 135:1087--1099.
  
\item Chandler, R.B., J. Hepinstall-Cymerman, S. Merker, H. Abernathy-Conners,
  and R.J. Cooper. 2018. Characterizing spatio-temporal 
  variation in survival and recruitment with integrated population
  models. {\it Special Collection on Integrated Population
    Models}. Auk 135:409--426. 

\item Hsiung, A.C., W.A. Boyle, R.J. Cooper, and
  R.B. Chandler. 2018. Altitudinal migration: ecological drivers,
  knowledge gaps, and conservation implications. Biological Reviews 
  93:2049--2070. 

\item Yeiser, J., J. Morgan, D. Baxley, R.B. Chandler, and
  J. Martin. 2018. Evidence that private land conservation
  has landscape-scale benefits for wildlife in agroecosystems. Journal
  of Applied Ecology 55:1930--1939. 

\item McFarland, K.P., J.D. Lloyd, S.J.K. Frey, P.L. Johnson,
  R.B. Chandler, and C.C. Rimmer. 2018. Modeling spatial
  variation in winter abundance to direct conservation actions for a
  vulnerable migratory songbird, Bicknell’s Thrush ({\it Catharus
    bicknelli}). The Condor 120:517--529. 

\item Howell, P.E., E. Muths, B. Hossack, B. Sigafus, and
  R.B. Chandler. 2018. Increasing connectivity between 
  metapopulation ecology and landscape ecology. Ecology 99:1119--1128.  
  
\item Ferrari, B.A., B.M. Shamblin, R.B. Chandler, H.R. Tumas, S. Hach\'e,
  L. Reitsma, and C.J. Nairn. 2018. Canada Warbler ({\it
  Cardellina canadensis}): Novel molecular markers and a preliminary
  analysis of genetic diversity and structure. Avian Conservation and
  Ecology 13:8 \url{https://doi.org/10.5751/ACE-01176-130108}.

\item Laufenberg, J.S., J.D. Clark, and
  R.B. Chandler. 2018. Estimating population extinction thresholds 
  with categorical classification trees for Louisiana black
  bears. PLoS ONE 13(1): e0191435.
  \url{https://doi.org/10.1371/journal.pone.0191435}

\item Cherry, M.J., R.B. Chandler, E.P. Garrison, D.A. Crawford,
  B.D. Kelly, D.B. Shindle, K. Godsea, K.V. Miller,
  L.M. Conner. 2018. Wildfire affects space use and movement of
  white-tailed deer in a tropical pyric landscape. Forest Ecology and
  Management 409: 161--169. 

\item Whittington, J., M. Hebblewhite, and
  R. Chandler. 2018. Generalized spatial mark-resight models with an 
  application to grizzly bears. Journal of Applied Ecology
  55:157--168. 

%\item[] {\centering 2017 \\}


\item[] {\centering 2016 \\}

\item Chandler, R.B. and J. Hepinstall-Cymerman. 2016. Estimating
  the spatial scales of landscape effects on abundance. Landscape
  Ecology 31:1383--1394. 

\item Howell, P.E., B.R. Hossack, E. Muths, B.H. Sigafus, and
  R.B. Chandler. 2016. Survival estimates of reintroduced populations
  of the Chiricahua leopard frog ({\it Lithobates
    chiricahuensis}). Copeia 4:824--830.   

\item Monroe, A., R.B. Chandler, W. Burger, Jr., and J.A. Martin. 2016. 
  Converting exotic forages to native warm-season grass can
  increase avian productivity in beef production systems. Agriculture,
  Ecosystems, and Environment 233:85--93. 

\item Dziadzio, M.C., L.L. Smith, R.B. Chandler, and
  S.B. Castleberry. 2016. Effect of nest location on gopher
  tortoise nest survival. Journal of Wildlife Management 80:1314--1322.

\item Laufenberg, J., J.D. Clark, M.J. Hooker, C.L. Lowe,
  K.C. O'Connell-Goode, J.C. Troxler, M.M. Davidson, M.J. Chamberlain,
  R.B. Chandler. 2016. Demographic rates and population viability
  of black bears in Louisiana. Wildlife Monographs 194:1--37.

\item Taylor, A.R., J.H. Schacke, T.R. Speakman, S.B. Castleberry,
  R.B. Chandler. 2016. Factors related to common bottlenose dolphin
  ({\it Tursiops truncatus}) seasonal migration along South Carolina
  and Georgia coasts, USA. Animal Migration 3:14--26.

\item Dziadzio, M.C., A.K. Long, L.L. Smith, R.B. Chandler, and
  S.B. Castleberry. 2016. Presence of red imported fire ants at
  gopher tortoise nests. Wildlife Society Bulletin 40:202--206.  

%\newpage

\item[] {\centering 2015 \\}

\item Chandler, R.B., E. Muths, B.H. Sigafus, C.R. Schwalbe,
  C.J. Jarchow, and B.R. Hossack. 2015. Spatial occupancy models for
  predicting metapopulation dynamics and viability following
  reintroduction. Journal of Applied Ecology 52:1325--1333.

\item Sollmann, R., B. Gardner, R.B. Chandler, J.A. Royle, and
  T.S. Sillett. 2015. An open population hierarchical
  distance sampling model. Ecology 96:325--331.

\item Hostetler, J.A. and R.B. Chandler. 2015. Improved
  state-space models for inference about spatial and temporal
  variation in abundance from count data. Ecology 96:1713--1723.

\item[] {\centering 2014 \\}

\item Chandler, R.B. and J.D. Clark. 2014. Spatially
  explicit integrated population models. Methods in Ecology and
  Evolution 5:1351--1360.

\item Graves, T., R.B. Chandler, J.A. Royle, P. Beier, and
  K.C. Kendall. 2014. Estimating landscape resistance to
  dispersal. Landscape Ecology 29:1201--1211.

\item Royle, J.A., R.B. Chandler, C.C. Sun, and A.K. Fuller. 2014. 
  Reply to Efford on   {\it ``Integrating Resource Selection
    Information with Spatial Capture-recapture.''} Methods in Ecology
  and Evolution 5:603--605. 

\item Zipkin, E.F., T.S. Sillett, E.H.C. Grant, R.B. Chandler, and
  J.A. Royle. 2014. Inferences about population dynamics
  from count data using multistate models: a comparison to
  capture-recapture approaches. Ecology and Evolution 4:417--426.

\item Zipkin, E.F., J.T. Thorson, K. See, H.J. Lynch, E.H.C. Grant,
  Y. Kanno, Y., R.B. Chandler, B.H. Letcher, and J.A. Royle.
  2014. Modeling structured population dynamics using
  data from unmarked individuals. Ecology 95:22--29.

%\newpage

\item[] {\centering 2013 \\}

\item Chandler, R.B., D.I. King, C.C. Chandler, R. Raudales,
  R. Trubey, and V. J. Arce Chavez. 2013. A small-scale
  land-sparing approach to conserving biodiversity in tropical
  agricultural landscapes. Conservation Biology 27:785--795. 
%  (Selected as Faculty of 1000 paper).

\item Sollmann, R., B. Gardner, R.B. Chandler, D. Shindle,
  D. Onorato, J.A. Royle, and A. O'Connell. 2013.
  Using multiple data sources provides density estimates for
  endangered Florida panther. Journal of Applied Ecology 50:961--968.

\item Chandler, R.B. and J.A. Royle. 2013.
  Spatially-explicit models for inference about density in unmarked
  populations. Annals of Applied Statistics 7:936--954.

\item Royle, J.A., R.B. Chandler, K.D. Gazenski, and
  T.A. Graves. 2013. Spatial capture-recapture models for jointly
  estimating population density and landscape connectivity. Ecology
  94:287--294. 

\item J.A. Royle, R.B. Chandler, C.C. Sun, and
  A.K. Fuller. 2013. Integrating resource selection information with
  spatial capture-recapture. Methods in Ecology and Evolution 4:520--530.

\item Yackulic, C.B., R.B. Chandler, E. Zipkin, J.A. Royle,
  J.D. Nichols, E.H.C. Grant, and S. Veran. 2013. Presence-only
  modeling using MAXENT: When can we trust the inferences? Methods in
  Ecology and Evolution 4:236--243. 

\item[] {\centering 2012 \\}

\item Sillett, T.S., R.B. Chandler, J.A. Royle, M. K\'{e}ry,
  S.A. Morrison. 2012. Hierarchical distance sampling models of
  habitat-specific abundance and total population size of the Island
  Scrub-Jay. Ecological Applications 22:1997--2006.

\item Royle, J.A, R.B. Chandler, C. Yackulic, and
  J.D. Nichols. 2012. Likelihood analysis of species occurrence
  probability from presence-only data for modeling species
  distributions. Methods in Ecology and Evolution 157:261--280.

\item Chandler, C.C., D.I. King, and R.B. Chandler. 2012. Do mature
  forest birds prefer early-successional habitat
  during the postfledging period? Forest Ecology and Management 264:1--9.

\item[] {\centering 2011 \\}

\item Chandler, R.B. and D.I. King. 2011. Habitat quality and
  habitat selection of golden-winged warbler warblers in Costa Rica:
  application of hierarchical models for open populations. Journal of
  Applied Ecology 48:1038--1047.

\item King, D.I., C.C. Chandler, J.H. Rappole, R.B. Chandler, and
  D.W. Mehlman. 2011. Establishing quantitative habitat targets
  for an endangered Neotropical migrant ({\it Dendroica chrysoparia})
  during the non-breeding season. Bird Conservation
  International. DOI: 10.1017/S095927091100027X.

\item Schlossberg, S., D.I. King, and R.B. Chandler. 2011. Effects
  of low-density housing development on shrubland birds
  in western Massachusetts. Landscape and Urban Planning 103:64--73.

\item Fiske I. and R.B. Chandler. 2011. unmarked: an R package
  for fitting hierarchical models of wildlife occurrence and
  abundance. Journal of Statistical Software 43:1--23.

\item Chandler, R.B., J.A. Royle, and D.I. King. 2011. Inference
  about density and temporary emigration in unmarked
  populations. Ecology 92:1429--1435.

%\newpage

\item[] {\centering 2010 \\}

\item Schlossberg, S., D.I. King, R.B. Chandler, and
  B.A. Mazzei. 2010. Regional synthesis of habitat relationships in
  shrubland birds. Journal of Wildlife Management 74:1513--1522.

\item[] {\centering 2009 \\}

\item Chavez-Arce, V.J.R. Raudales, R. Trubey, D.I. King,
  R.B. Chandler, and C. Chandler. 2009. Measuring and managing the
  environmental costs of coffee production in Latin
  America. Conservation and Society 7:141--144.

\item Wolfe, J.D., R.B. Chandler, and D.I. King. 2009. Molt
  patterns, age, and sex criteria for selected highland Costa Rica
  landbirds. Ornitholog\'{i}a Neotropical 20:451--459.

\item Chandler, R.B., D.I. King, and C.C. Chandler. 2009. Effects
  of management regime on the abundance and nest survival of shrubland
  birds in wildlife openings in northern New England, USA. Forest
  Ecology and Management 258:1669--1676.

\item Chandler, R.B., King, D.I., and
  S. DeStefano. 2009. Scrub-shrub bird habitat associations at
  multiple spatial scales in beaver meadows in Massachusetts. Auk 126:
  186--197.

\item King, D.I., R.B. Chandler, S. Schlossberg, and
  C.C. Chandler. 2009. Habitat use and nest success of scrub-shrub
  birds in wildlife and silvicultural openings in western
  Massachusetts, U.S.A. Forest Ecology and Management 257:421--426.

\item King, D.I., R.B. Chandler, J.M. Collins, W.R. Peterson, and
  T.E. Lautzenheiser. 2009. Effects of width, edge and habitat on the
  abundance and nesting success of scrub-shrub birds in powerline
  corridors. Biological Conservation 142:2672--2680.

\item[] {\centering 2004 \\}

\item Chandler R.B., A.M. Strong, and C.C. Kaufman. 2004. Elevated
  lead levels in urban house sparrows: A threat to sharp-shinned hawks
  and merlins? Journal of Raptor Research 38:62--68.

\end{itemize}



%\newpage %%%%%%% !!!!!!





\begin{comment}
\emph{Technical Reports}
\begin{itemize}
  \item Chandler, R.B., D.I. King, and C.C. Chandler. 2006. Butterfly
    occurrence and species richness in wildlife openings and clearcuts
    on the White Mountain National Forest. A Report to the Wildlife TES
    Program, White Mountain National Forest.
\end{itemize}
\end{comment}

% \newpage

\emph{Book chapters}
\begin{itemize}
  \item Chandler, R.B., S. Tolfree, J. Gerwin, C. Smalling,
    L. Chavarr\'ia-Duriaux, G. Duriaux, and
    D.I. King. 2015. Conservation implications of golden-winged 
    warbler social and  
    foraging behaviors during the nonbreeding season. {{\it In}
      Golden-winged Warbler Ecology, Conservation, and Management}
    (Streby, H.M., D.E. Anderson, and D.A. Buehler, eds.), Studies in
    Avian Biology 49:175--192. CRC Press. 

  \item King, D.I., R.B. Chandler, C. Smalling, R. Trubey,
    R. Raudales, and T. Will. 2016. Nonbreeding golden-winged
    warbler habitat: status, conservation, and needs. {{\it In} Golden-winged
    Warbler Ecology, Conservation, and Management} (Streby, H.M.,
    D.E. Anderson, and D.A. Buehler, eds.), Studies in Avian Biology
    29:29--38. CRC Press. 

  \item Rosenberg, K.V., T. Will, D.A. Buehler, S.B. Swarthout,
    W.E. Thogmartin, and R.B. Chandler. 2016. Historic and current
    distribution and population status of golden-winged warblers. {\it
      In} Golden-winged Warbler Ecology, Conservation, and Management
    (Streby, H.M., D.E. Anderson, and D.A. Buehler, eds.), Studies in
    Avian Biology 49:3--28. CRC Press. 

  \item King. D.I., R.B. Chandler, J.H. Rappole, R. Raudales, and
    R. Trubey. 2012. Community-Based
    Agroforestry Initiatives in Nicaragua and Costa Rica for
    Biodiversity Conservation. Pages 99--115 {\it In} Biodiversity
    conservation in agroforestry landscapes: challenges and
    opportunities. Javier Simonetti, Audrey A. Grez Cristi\'{a}n,
    F. Estades and Jorge P\'{e}rez, \emph{Eds}. Universidad de Chile Press.
\end{itemize}


\vspace{0.5cm}

%\newpage

{\large \bf Presentations} \\
\rule[3mm]{\textwidth}{0.3mm}

\emph{Invited -- seminars}

\begin{itemize}

\item Chandler, R.B. 2019. Dynamics of Trailing-edge Bird Populations
  in the Southern Appalachian Mountains. Department of Fish and
  Wildlife Conservation, Virginia Tech.

\item Chandler, R.B. 2019. Dynamics and Viability of Trailing-edge
  Populations in the Southern Appalachian Mountains. Odum School of
  Ecology, University of Georgia.

\item Chandler, R.B. 2018. Narrowing the divide between ecological
  theory and application with hierarchical spatio-temporal
  models. Department of Wildlife Ecology and Conservation, 
  University of Florida.

\item Chandler, R.B. 2016. Narrowing the divide between ecological
  theory and practice with hierarchical spatial models. Department of
  Forest and Wildlife Ecology, University of Wisconsin Madison. 

\item Chandler, R.B. and P.E. Howell. 2015. Integrating landscape
  resistance into spatial metapopulation models. Computational Ecology
  and Epidemiology Study Group. Odum School of Ecology, University of
  Georgia. 

\item Chandler, R.B. 2015. Spatial correlation as information about
  ecological processes. Warnell Schoool of Forestry, University of
  Georgia. 

\item Chandler, R.B. 2014. The role of spatial models in applied
  ecological research. Auburn University.

\item Chandler, R.B. 2012. Land-sparing coffee production systems for avian
  conservation. Smithsonian Migratory Bird Center. 

\item Chandler, R.B., 2012. Population dynamics and species distributions following
  secondary contact. University of New Hampshire.

\end{itemize}

\emph{Invited -- conferences}

\begin{itemize}

\item Chandler, R.B., M.J. Cherry, K.V. Miller, L.M. Conner,
  R.J. Warren, E. Garrison, D. Shindle, D. Crawford, B. Kelly, and K. 
  Engebretsen. 2016. Efficient monitoring of unmarked prey populations
  by combining camera and telemetry data. The 23rd Annual Meeting of
  The Wildlife Society. Raleigh, NC.

\item Chandler, R.B., R.J. Cooper, J. Hepinstall-Cymerman, S. Merker, R.
  Chitwood, H. Abernathy. 2016. Spatially explicit integrated
  population models for understanding the mechanisms governing range
  shifts. North American Ornithological Conference, Washington DC. 

\item Chandler, R.B. 2015. Modeling population dynamics using data
  from marked and unmarked individuals. 22nd Annual Conference of The
  Wildlife Society. Winnipeg, Manitoba.

\item Chandler, R.B. 2015. Estimating Deer Density by Combining Camera
  and Telemetry Data. Annual meeting of the Georgia Chapter of The
  Wildlife Society. Athens, GA.

\item Chandler, R.B. and S.A. Merker. 2015. Integrated population
  models for predicting spatial and temporal dynamics at range
  margins. 100th Annual Conference of the Ecological Society of
  America. Baltimore, MD. 

\item Chandler, R.B and J.D. Clark. 2013. Cost-efficient monitoring
  using capture-recapture methods. The Georgia Chapter Meeting of the
  Wildlife Society. Athens, GA.

\item Chandler, R.B., J.A. Royle, J. Sauer, and
  K. Pardieck. 2012. Modeling population dynamics and detection
  probability using Breeding Bird Survey data. 18th Annual Meeting of
  the Wildlife Society. Waikoloa, HI.

\item Chandler, R.B. 2011. Modeling population dynamics and detection
  probability using Breeding Bird Survey data. The International
  Environmetrics Society's Third North
  American Regional Meeting. La Crosse, WI.

\item Chandler, R.B., D.I. King, and K.V. Rosenberg. 2010. Monitoring
  Neotropical-Nearctic migratory birds during the non-breeding
  season. Midwest Fish and Wildlife Conference. Minneapolis, MN.

\end{itemize}

%\newpage % XXXXX !!!!!!

\emph{Contributed}

\begin{itemize}

% \item Cherry, M. J., E. Garrison, R. B. Chandler, D. Shindle, C. Morea,
%   L. M. Conner, R. J.  Warren, K. V.  Miller.  Ungulate Population
%   Fluctuations in South Florida: Predators, Fire, and Floods. 69th
%   Conference of the Southeastern Association of Fish and Wildlife
%   Agencies, Asheville, NC, USA.  

\item Chandler, R.B., E. Muths, B. Sigafus, C. Schwalbe, C. Jarchow,
  and B. Hossack. 2014. Spatial occupancy models for metapopulation
  viability analysis. The International Statistical Ecology
  Conference. Montpellier, France.

\item Chandler, R.B. and D.I. King. 2009. Golden-winged Warbler
  habitat selection in Costa Rica. 127th Stated Meeting of the
  American Ornithologists’ Union. Philadelphia, PA.

\item Chandler, R.B., D.I. King, and
  C.C. Chandler. 2008. Conservation implications of Golden-winged
  Warbler non-breeding ecology in a forest-agriculture mosaic in Costa
  Rica. 126th Stated Meeting of the American Ornithologists’
  Union. Portland, OR.

\item Chandler, R.B., D.I. King, V.J. Chavez A., C.C. Chandler,
  R. Raudales, and R. Trubey. 2007. Species richness, composition, and
  body condition in a novel coffee cultivation system ``Integrated Open
  Canopy" in Costa Rica. 125th State Meeting of the American
  Ornithologists’ Union. Laramie, WY.

% \item King, D.I., C.C. Chandler, R.B. Chandler, and
%   R.M. DeGraaf. 2005. Effects of silviculture on mature forest and
%   early-successional shrubland passerine birds in Northern and Central
%   New England. New England Society of American Foresters, Portland,
%   ME.

% \item King, D.I., C.C. Chandler, R.B. Chandler, and
%   R.M. DeGraaf. 2005. Effects of forestry on birds. New Hampshire
%   State Lands Management Team, Concord, NH.

\item Chandler R.B., D.I. King, and C.C. Chandler. 2005. Effects of
  patch size and treatment method on early-successional birds in
  managed shrublands in New Hampshire. 123rd State Meeting of the
  American Ornithologists’ Union meeting, Santa Barbara, CA.

\item Chandler. C.C., D.I. King and R.B. Chandler. 2005. Habitat use
  of Neotropical migrant birds during the pre-migratory period.
  123rd Stated Meeting of the American Ornithologists Union, Santa
  Barbara, CA.

\end{itemize}




% \vspace{0.5cm}

% {\large \bf Invited Research Seminars} \\
% \rule{\textwidth}{0.3mm}

% \begin{itemize}

%   \item Smithsonian Migratory Bird Center. April,
%     2012. Land-sparing coffee production systems for avian conservation.

%   \item University of New Hampshire. November,
%     2012. Population dynamics and species distributions following
%     secondary contact.

% \end{itemize}



%\vspace{0.5cm}
%\newpage

{\large \bf Grants} \\
\rule[3mm]{\textwidth}{0.3mm}

\begin{itemize}

\item Effects of Working Lands for Wildlife and Pine Savanna
  Conservation Projects on Northern Bobwhite. 2019. Co-PI with
  Dr. James Martin. Pheasants Forever, Inc. \$230,000

\item Central Georgia Black Bear Population Viability Analysis
  (PVA). 2019. Co-PI with Dr. Michael Chamberlain. Georgia Department
  of Natural Resources. \$85,000. 
  
\item Individual-based species distribution models for
  understanding the demographic drivers of range shifts and their
  consequences for biodiversity. 2017. PI. National Science
  Foundation Faculty Early Career Development Award. \$719,862. 

\item Migratory connectivity of wintering golden-winged
  warblers. 2016. PI. Subcontract from Indiana University of
  Pennsylvania. \$13,200. 

\item Improving northern bobwhite abundance estimates to set harvest
  regulations. 2015. Co-PI with Dr. James Martin. Georgia Department of
  Natural Resources. \$285,942.

%\item Impacts of rabbit hunting

\item Habitat loss and fragmentation effects on northern bobwhites and
  eastern meadowlarks. 2015. Co-PI with Dr. James Martin, Dr. Clint Moore,
  Dr. Jim Giocomo, and Dr. Myung-Bok Lee. Gulf Coast CESU. \$50,000. 

\item Effects of conservation reserve enhancement program on bird
  populations at local landscape scales in Kentucky. 2015. Co-PI with
  Dr. James Martin. Kentucky Department of Natural Resources. \$165,997.

\item Population dynamics and viability of white-tailed deer and
  Florida panther in a rapidly changing environment. 2014. PI with Dr. Karl
  Miller, Dr. Robert Warren, and Dr. Mike Conner. Florida
  Fish and Wildlife Commission. \$1,162,602. 

\item Refining post-delisting monitoring for Louisiana black
  bears. 2014. PI with Dr. Joseph Clark. Louisiana Department of Wildlife and
  Fisheries. Subcontract through University of Tennessee. \$82,000. 

\item Development and evaluation of an unbaited camera survey
  technique for estimating demographic parameters of white-tailed
  deer. 2014. Co-PI with Dr. Karl Miller. \$374,000.  

\item Integrating metapopulation ecology and landscape ecology for
  improved population viability analysis and conservation
  decision-making. 2013. Co-PI with Dr. Blake Hossack, Dr. Erin Muths, and
  Dr. Brian Irwin. U.S. Geological survey. \$210,000.

\item Estimating population size and viability of the Central Georgia
  black bear population. 2013. Co-PI with Dr. Michael
  Chamberlain. Georgia Department of Natural
  Resources. 2014--2016. \$366,000. 

\item Analysis of golden-winged warbler winter distribution
  data. 2011. Cornell Lab of Ornithology. \$12,000. 

\item Development of the {\bf R} Package {\tt unmarked} to Analyze
  Data from Long-term Monitoring Programs. 2011--2012. National Park
  Service. \$50,000. 

\item Habitat Selection and Habitat-specific Survival of Wintering
  Golden-winged Warblers {\it (Vermivora chrysoptera)} in Managed Landscapes
  in Costa Rica. 2009--2010. USDA Forest Service International
  Programs. \$64,000. 

\item Migratory Bird Conservation Using Alternative Coffee
  Cultivation and Processing Methodologies. 2006--2008.  US Fish and
  Wildlife Service, Division of Bird Habitat Conservation. (with
  the Mesoamerican Development Institute). \$137,000.

\end{itemize}


\vspace{0.5cm}

%\newpage % !!!!!!!!!!


{\large \bf Awards and Honors} \\
\rule[3mm]{\textwidth}{0.3mm}

\begin{itemize}

\item Recipient of the Warnell School of Forestry and Natural
  Resources' Alumni Association Faculty Award for Early Career
  Teaching. 2019. 
  
\item Recipient of the University of Georgia's Fred C. Davidson Early
  Career Scholar Award. 2017. 

\item Elected Treasurer of the Wildlife Society's Biometrics Working
  Group. 2015. 

\item Elected to the board of the Wildlife Society's Biometrics
  Working Group. 2014. 

\item Named Elective Member of the American Ornithologists' Union,
  2013. 

\item Co-recipient of the Wildlife Society's 2013 spatial ecology award for
  the development of the {\bf R} package {\tt unmarked}.

\item Co-recipient of the 2009 US Forest Service ``Wings Across the Americas Award''
for outstanding research achievement in protection of
migratory birds and their habitat.

\item American Ornithologists' Union Student Travel Award. June 2007.

\item Dean's Junior Book Award for outstanding academic achievement
  and leadership potential. University of Vermont. May 2001.

\item Office of Sponsored Programs, University of Vermont award for
  outstanding undergraduate research. October 2000.

\end{itemize}


\vspace{0.5cm}
%\newpage              %%% NOTE






{\large \bf Teaching} \\
\rule[3mm]{\textwidth}{0.3mm}

{\it Courses at UGA}

\begin{itemize}

\item Estimation of Fish and Wildlife Population Parameters. WILD/FISH
  8390. Co-instructed with Dr. Mike Conroy. (Spring 2017--2018)

\item Experimental Design and Analysis. FANR 6750. Co-instructed with
  Dr. Robert Cooper. (Fall 2013--2018)

\item Applied Population Dynamics. WILD 5700/7700. (Spring 2014--2019)

\item Spatial Capture-Recapture. WILD 8300. (Spring 2015 \& 2016)

\item Conserving Wildlife in Agriclutural Landscapes. WILD
  5300/7300. (Fall, 2014) 

\end{itemize}

\vspace{0.5cm}

{\it Workshops}

\begin{itemize}

\item Spatial capture-recapture workshop. Oct 2016. The Wildlife
  Society Conference, Raleigh NC. With J.A. Royle and C. Sutherland. 

\item Spatial capture-recapture workshop. July 2016. University of
  Georgia. With J.A. Royle and B. Augustine.

\item Spatial capture-recapture workshop. March 2015. University of
  Georgia. With J.A. Royle and J. Laufenberg.

\item Hierarchical models for abundance, distribution
  and species richness in spatially-structured populations using
  \texttt{unmarked} and \textbf{WinBUGS}. Workshop at Patuxent Wildlife.
  Nov 2012 with M. K\'{e}ry and J.A. Royle.

\item Estimating avian abundance and occupancy with marked and
  unmarked individuals. Workshop given at the 5th North American
  Ornithological Conference, Vancouver, British Columbia,
  Canada in August 2012. With E. Cooch, P. Doherty, and J.D. Nichols.

\item Hierarchical modeling using the R package
  \textbf{unmarked}. Workshop given at International Statistical
  Ecology Conference with J.A. Royle. July 2012.
  \url{http://www.cees.uio.no/isec2012/workshops/unmarked.html}

\item Hierarchical models for abundance, distribution
  and species richness in spatially-structured populations using
  \texttt{unmarked} and \textbf{WinBUGS}. Workshop at Patuxent Wildlife.
  April 2012 with M. K\'{e}ry and J.A. Royle.

\item Analysis of wildlife occurrence and abundance data using the \textbf{R}
  package \textbf{unmarked}. January 2012. With J.A. Royle.
  \url{https://sites.google.com/site/unmarkedinfo/home/webinars/2012-january}

\item Hierarchical models of wildlife occurrence, abundance, and
  survival. 2011. Graduate-level course taught at
  UMass. With W. Deluca. \\
  \url{https://sites.google.com/site/hierarchicalmodelingcourse/}

\item Monitoring Nearctic birds in the Neotropics. 2008. Field course
  taught to Latin American researchers in Colombia.

\end{itemize}



%\vspace{0.5cm}
%\newpage

% {\large \bf Postgraduate Training} \\
% \rule{\textwidth}{0.3mm}

% \begin{itemize}

% \item Bayesian Population Analysis. November, 2010. Instructed by Marc
%   K\'{e}ry and Michael Schaub.

% \item Adaptive Management for Recurrent Decisions. May,
%   2012. Instructed by Mike Runge, Jim Nichols, Scott Boomer, Clint
%   Moore, and Jill Gannon.

% \end{itemize}



\vspace{0.5cm}

%\newpage  % !!!!!!

{\large \bf Referee Service} \\
\rule[3mm]{\textwidth}{0.3mm}

\begin{itemize}
  \item[] Animal Conservation, Annals of Applied Statistics, The Auk,
    Biometrics, Canadian Journal of Fisheries and Aquatic Sciences,
    Diversity and Distributions, Ecography, Ecological Applications,
    Ecology, Ecology Letters, Folia Zoologica, Freshwater Biology,
    Ibis, Journal of Applied Ecology, Journal of Mammology, Methods in
    Ecology and Evolution, Nature Communications, Ornotholog\'{i}a
    Neotropical, PLoS One. 
\end{itemize}



%\newpage

\vspace{0.5cm}

{\large \bf Professional Memberships} \\
\rule[3mm]{\textwidth}{0.3mm}

\begin{itemize}
\item[] Ecological Society of America, American Ornithological 
  Society (Elective Member), The Wildlife Society
\end{itemize}


%\newpage



\vspace{0.5cm}

{\large \bf Advising} \\
\rule[3mm]{\textwidth}{0.3mm}

{\it Graduate students}
\begin{itemize}
  \item Samuel Merker, MS candidate. Investigating the factors
    limiting species distributions at low latitude range margins. 
  \item James Johnson, PhD candidate (co-advised with Dr. Karl
    Miller). Monitoring white-tailed deer populations using camera
    traps. 
  \item Lydia Stiffler, PhD candidate. Spatial population dynamics of
    white-tailed deer in the Florida panther range.
  \item Tori Mezebish. MS candidate. Overwinter survival and migratory
    behavior of ring-necked ducks.
\end{itemize}

{\it Postdoctoral researchers}
\begin{itemize}
   \item Florent Bled. Monitoring white-tailed deer and Florida panther
     population dynamics. Co-sponsored with Dr. Mike Conner,
     Dr. Robert Warren, and Dr. Karl Miller
   \item Greg Wann. Population dynamics of ruffed grouse near the
     southern extent of their range.  
\end{itemize}

%\newpage

{\it Former graduate students}
\begin{itemize}
%  \item Brian Kelly, MS student (co-advised with Dr. Karl
%    Miller). Cause-specific mortality of white-tailed deer in South
%    Florida.
  \item Ryan Chitwood, MS (co-advised with Dr. Robert
    J. Cooper). Population dynamics of the black-throated blue warbler. 
  \item Kristin Engebretsen, MS (co-advised with Dr. Karl
    Miller). White-tailed deer recruitment in the Florida panther
    range.
  \item Michael Biggerstaff, MS (co-advised with Dr. Karl
    Miller). Activity patterns of white-tailed deer in southern Georgia. 
  \item Paige Howell, PhD. Spatial metapopulation models for 
    informing conservation decisions.
  \item Joel Owen, MS. Nest site selection of red-and-green macaws. 
  \item An Chee Hsiung, MS. Conservation in tropical agricultural landscapes.
  \item Daniel Crawford, MS (co-advised with Dr. Karl
    Miller). Resource selection of white-tailed deer in South
    Florida.
  \item Jared Green, MS. (Co-advised with Dr. Tracey
    Tuberville). Effectiveness of head-starting as a management tool
    for establishing a viable population of blanding's turtles. 
\end{itemize}

%\newpage



{\it Former postdoctoral researchers}
\begin{itemize}
   \item Michael Cherry. Monitoring white-tailed deer and Florida panther
     population dynamics. Co-sponsored with Dr. Mike Conner,
     Dr. Robert Warren, and Dr. Karl Miller
   \item Jared Laufenberg. Refinements in
     monitoring methods for the Louisiana Black Bear. Co-sponsored
     with Dr. Joseph Clark.  
\end{itemize}

% {\it Senior thesis students}
% \begin{itemize}
%   \item Erin Daughtrey
%   \item Tyler Gagat
%   \item \dots
% \end{itemize}

% {\it Committee service}
% \begin{itemize}
%   \item Jeffrey Ritterson. M.S. Candidate, University of Massachusetts
%     Amherst. %Habitat-specific survival of the
%     %Golden-winged Warbler in Costa Rica.
%   \item Jared Green. Ph.D. candidate, University of Georgia.
%   \item Kelsey Turner. M.S. candidate, University of Georgia.
%   \item David Stone. Ph.D. candidate, University of Georgia.
%   \item Kellie Phillips. Ph.D. candidate, University of Georgia.
%   \item Sarah Coker. Ph.D. candidate, University of Georgia.
% \end{itemize}




